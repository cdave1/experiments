\documentclass[10pt,twoside,a4paper]{article}
\usepackage[dvips]{graphicx}
\usepackage{verbatim}
\usepackage{amsmath}
\author{David Petrie}
\title{Ruby on Rails Notes}
\begin{document}
\maketitle


\section{Rails Environment}

\subsection{Emacs as Ruby-on-Rails IDE}

Install the following plugins for emacs:
\begin{itemize}
\item RHTML and Rinari
\item find-recursive and snippet.el
\item emacs-rails
\item ruby-electic.el
\end{itemize}

Most of this stuff available on rubyforge, though will need to google for ruby-electric.el.

\subsection{Creating a Rails Project}

\begin{verbatim}
$ rails <project_name>
\end{verbatim}

After that, all the initial help you need is to be found in the README file in the resulting ruby application.

\section{The Ruby Language}

Created by Yukihiro Matsumoto (aka matz).

\subsection{Notes}

This subsection contains various bits of pieces from the web. Not including attribution so I can do this quickly, but needless to say most of the notes in this subsection were not written by me.

\subsubsection{Pass-By-Ref or Pass-By-Value}

Ruby only has objects, so pass by 'value' is nonsensical, there are only references. Consequently Ruby has no 'simple' types.

However, watch for methods that return new object. int + int returns a new int, it doesn't modify the value of the old int.

In C, when you have "int a = 1;", the value of a is 1 (a primitive type). In Ruby, when you have "a = 1", the value of a is actually a reference to an instance of Fixnum which represents the integer 1. Ruby has no primitive types (ok, it's implimented in C, so they do exist behind the scenes; but there is no way to access a pure primitive type within your Ruby code).

In the following Ruby code:

  a = 1
  b = a

the first line reserves memory space (behind the scenes in the code for the Ruby interpreter) for an instance of Fixnum and creates a new instance of Fixnum with the value of 1. It then assigns a reference to that instance of Fixnum to the variable a. However, the second line does not create a new Fixnum address space in memory. The second line simply tells the variable b to reference the same object that variable a currently references. This is similar to (but not exactly the same as) the following C code:

  int x = 1; /* this (sort of) happens behind the scenes in Ruby */
  int a = &x;
  int b = a;

The difference is that, since Ruby doesn't have explicit pointers, any methods invoked on the a or b variables are automatically dereferenced and called on the actual Fixnum object those variable point to (I guess you could say the "pointer" is both implicit _and_ unavoidable). In C, you would have to specifically dereference the variable, otherwise you would be operating on the memory address instead of the contents of that memory space.






\subsection{Commenting}

\begin{verbatim}
\end{verbatim}

\begin{verbatim}
# comment
\end{verbatim}

or, for comment blocks:

\begin{verbatim}
=begin
comment line 1
comment line 2
etc...
=end
\end{verbatim}



\subsection{Running}

Interactive mode:

\begin{verbatim}
$ irb
irb(main):001:0> 1+1
=> 2
irb(main):002:0> print "helloworld"
helloworld=> nil
\end{verbatim}

A script file:

\begin{verbatim}
$ touch myprogram.rb
$ echo "print \"helloworld\n\"" >> myprogram.rb
$ ruby myprogram.rb
\end{verbatim}

As far as I can tell, ruby doesn't really do compilation, though there is a strange ruby to python compiler (see $http://github.com/why/unholy/tree/master$).



\subsection{Lists}

Basic array declaration:

\begin{verbatim}
myarray = [1,2,2,3,5,6,7]
\end{verbatim}


\subsection{File IO}

Reading:

\subsection{Control Structures}

Loops

\begin{verbatim}
mylist = [1,2,2,3,5,6,7]
mylist.each do |item|
  print "%d\n" % (item * 3)
end


\end{verbatim}


\subsection{Functions}

map (except you map a block to some code), weirdly.

Lambda style
\begin{verbatim}
c = Proc.new { |v| v > 5 }
res = c.call(4)
\end{verbatim}

The items between the pipes in the block contain the ``input'' variables for the lambda statement.

\subsection{Files}

Write to a file:

\begin{verbatim}
tosort = []
count = 3000
while count > 0
  tosort << rand(1000)

  count -= 1
end

f = File.new("out", "w")

f.write(tosort.join("\n"))
\end{verbatim}

Read from a file:

\begin{verbatim}
f2 = File.new("out", "r")
fileList = f2.readlines()

puts fileList
\end{verbatim}



\end{document}
